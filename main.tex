% Bursa Teknik Üniversitesi, Mühendislik ve Doğa Bilimleri Fakültesi, Latex
% tez taslağı için giriş noktası.
%
% Bu proje MIT lisansı ile lisanslanmıştır.

% Yazarlar:
% * Ceyhun Şen <19360859023@ogrenci.btu.edu.tr>

% Sınıf dosyasını yükle.
\documentclass{btu_mdbf_tez}

% Genel tez bilgisi. Bu bilgileri kendi bilgileriniz ile değiştirin.
\title{BTÜ Latex Tez Taslağı}
\author{Öğrenci Adı-Soyadı}
\date{\today}

% Dökümanın giriş noktası.
\begin{document}
	% Tezin ilk ayarlarını yap.
	\startthesis

	% İç kapak sayfasını oluştur.
	\innercoverpage
		{Bölüm Adı}
		{Danışman Adı-Soyadı}

	% Beyan sayfasını oluştur.
	\declaration
		{Bölüm Adı}
		{123456789}
		{Danışman Adı-Soyadı}
		{Jüri Üyesi Adı-Soyadı \#1}
		{Üniversitesi}
		{Jüri Üyesi Adı-Soyadı \#2}
		{Üniversitesi}
		{Bölüm Başkanı Adı-Soyadı}

	% İçeriğin başlangıcı.
	\startcontent
	\chapter{Örnek Bölüm 1}

Bölüm Bölüm Bölüm Bölüm Bölüm Bölüm Bölüm Bölüm
Bölüm Bölüm Bölüm Bölüm Bölüm Bölüm Bölüm Bölüm
Bölüm Bölüm Bölüm Bölüm Bölüm Bölüm Bölüm Bölüm
Bölüm Bölüm Bölüm Bölüm Bölüm Bölüm Bölüm Bölüm
Bölüm Bölüm Bölüm Bölüm Bölüm Bölüm Bölüm Bölüm

\section{Örnek Başlık 1}

Başlık Başlık Başlık Başlık Başlık Başlık Başlık Başlık
Başlık Başlık Başlık Başlık Başlık Başlık Başlık Başlık
Başlık Başlık Başlık Başlık Başlık Başlık Başlık Başlık
Başlık Başlık Başlık Başlık Başlık Başlık Başlık Başlık
Başlık Başlık Başlık Başlık Başlık Başlık Başlık Başlık

\subsection{Örnek Alt Başlık 1}

Alt Başlık Alt Başlık Alt Başlık Alt Başlık Alt Başlık Alt Başlık
Alt Başlık Alt Başlık Alt Başlık Alt Başlık Alt Başlık Alt Başlık
Alt Başlık Alt Başlık Alt Başlık Alt Başlık Alt Başlık Alt Başlık
Alt Başlık Alt Başlık Alt Başlık Alt Başlık Alt Başlık Alt Başlık
Alt Başlık Alt Başlık Alt Başlık Alt Başlık Alt Başlık Alt Başlık

\subsubsection{Örnek Alt Alt Başlık 1}

Alt Alt Başlık Alt Alt Başlık Alt Alt Başlık Alt Alt Başlık Alt Alt Başlık
Alt Alt Başlık Alt Alt Başlık Alt Alt Başlık Alt Alt Başlık Alt Alt Başlık
Alt Alt Başlık Alt Alt Başlık Alt Alt Başlık Alt Alt Başlık Alt Alt Başlık
Alt Alt Başlık Alt Alt Başlık Alt Alt Başlık Alt Alt Başlık Alt Alt Başlık
Alt Alt Başlık Alt Alt Başlık Alt Alt Başlık Alt Alt Başlık Alt Alt Başlık

\section{Örnek Başlık 2}

Başlık Başlık Başlık Başlık Başlık Başlık Başlık Başlık
Başlık Başlık Başlık Başlık Başlık Başlık Başlık Başlık
Başlık Başlık Başlık Başlık Başlık Başlık Başlık Başlık
Başlık Başlık Başlık Başlık Başlık Başlık Başlık Başlık
Başlık Başlık Başlık Başlık Başlık Başlık Başlık Başlık

	\chapter{Örnek Bölüm 2}

Bölüm Bölüm Bölüm Bölüm Bölüm Bölüm Bölüm Bölüm
Bölüm Bölüm Bölüm Bölüm Bölüm Bölüm Bölüm Bölüm
Bölüm Bölüm Bölüm Bölüm Bölüm Bölüm Bölüm Bölüm
Bölüm Bölüm Bölüm Bölüm Bölüm Bölüm Bölüm Bölüm
Bölüm Bölüm Bölüm Bölüm Bölüm Bölüm Bölüm Bölüm

\section{Örnek Başlık 1}

Başlık Başlık Başlık Başlık Başlık Başlık Başlık Başlık
Başlık Başlık Başlık Başlık Başlık Başlık Başlık Başlık
Başlık Başlık Başlık Başlık Başlık Başlık Başlık Başlık
Başlık Başlık Başlık Başlık Başlık Başlık Başlık Başlık
Başlık Başlık Başlık Başlık Başlık Başlık Başlık Başlık

\subsection{Örnek Alt Başlık 1}

Alt Başlık Alt Başlık Alt Başlık Alt Başlık Alt Başlık Alt Başlık
Alt Başlık Alt Başlık Alt Başlık Alt Başlık Alt Başlık Alt Başlık
Alt Başlık Alt Başlık Alt Başlık Alt Başlık Alt Başlık Alt Başlık
Alt Başlık Alt Başlık Alt Başlık Alt Başlık Alt Başlık Alt Başlık
Alt Başlık Alt Başlık Alt Başlık Alt Başlık Alt Başlık Alt Başlık

\subsubsection{Örnek Alt Alt Başlık 1}

Alt Alt Başlık Alt Alt Başlık Alt Alt Başlık Alt Alt Başlık Alt Alt Başlık
Alt Alt Başlık Alt Alt Başlık Alt Alt Başlık Alt Alt Başlık Alt Alt Başlık
Alt Alt Başlık Alt Alt Başlık Alt Alt Başlık Alt Alt Başlık Alt Alt Başlık
Alt Alt Başlık Alt Alt Başlık Alt Alt Başlık Alt Alt Başlık Alt Alt Başlık
Alt Alt Başlık Alt Alt Başlık Alt Alt Başlık Alt Alt Başlık Alt Alt Başlık

\section{Örnek Başlık 2}

Başlık Başlık Başlık Başlık Başlık Başlık Başlık Başlık
Başlık Başlık Başlık Başlık Başlık Başlık Başlık Başlık
Başlık Başlık Başlık Başlık Başlık Başlık Başlık Başlık
Başlık Başlık Başlık Başlık Başlık Başlık Başlık Başlık
Başlık Başlık Başlık Başlık Başlık Başlık Başlık Başlık

\end{document}
